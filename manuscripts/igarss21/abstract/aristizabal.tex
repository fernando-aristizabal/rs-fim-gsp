% Template for IGARSS-2020 paper; to be used with:
%          spconf.sty  - LaTeX style file, and
%          IEEEbib.bst - IEEE bibliography style file.
% --------------------------------------------------------------------------
\documentclass{article}
\usepackage{style/spconf,amsmath,epsfig}

% Example definitions.
% --------------------
\def\x{{\mathbf x}}
\def\L{{\cal L}}

% Title.
% ------
\title{Riverine Flood Inundation Mapping with Graph Signal Filtering of Stages Determined from Unsupervised Clustering of Synthetic Aperture Rader Imagery}
%
% Single address.
% ---------------
%\name{Author(s) Name(s)\thanks{Thanks to XYZ agency for funding.}}
%\address{Author Affiliation(s)}
%
% For example:
% ------------
%\address{School\\
%	Department\\
%	Address}
%
% Two addresses (uncomment and modify for two-address case).
% ----------------------------------------------------------
\twoauthors
  {Fernando Aristizabal\sthanks{Thanks to National Oceanic and Atmospheric Administration (NOAA) Office of Water Predition (OWP)}}
    {Lynker Technologies \\
    NOAA-OWP Affiliate\\
    National Water Center\\
    205 Hackberry Lane, Tuscaloosa, AL 35401\\
    fernando.aristizabal@noaa.gov}
  {Jasmeet Judge\sthanks{The fourth author contributed while at University of Florida}}
    {University of Florida\\
    Center for Remote Sensing\\
    Agricultural and Biological Engineering\\
    1741 Museum Rd, Gainesville, FL 32611\\
    jasmeet@ufl.edu}
%
\begin{document}
%\ninept
%
\maketitle
%
\begin{abstract}
abstract
\end{abstract}
%
\begin{keywords}
flood inundation mapping, remote sensing, synthetic aperture radar, unsupervised learning, graph signal processing, height above nearest neighbor 
\end{keywords}
%
\section{Introduction}
\label{sec:intro}



\section{Materials and Methods}
\label{sec:materials_and_methods}

Materials and Methods

\section{Results and Discussion}
\label{sec:results_and_discussion}

Results and Discussion

% Below is an example of how to insert images. Delete the ``\vspace'' line,
% uncomment the preceding line ``\centerline...'' and replace ``imageX.ps''
% with a suitable PostScript file name.
\begin{figure}[htb]

\begin{minipage}[b]{1.0\linewidth}
  \centering
% \centerline{\epsfig{figure=image1.ps,width=8.5cm}}
  \vspace{2.0cm}
  \centerline{(a) Result 1}\medskip
\end{minipage}
%
\begin{minipage}[b]{.48\linewidth}
  \centering
% \centerline{\epsfig{figure=image3.ps,width=4.0cm}}
  \vspace{1.5cm}
  \centerline{(b) Results 3}\medskip
\end{minipage}
\hfill
\begin{minipage}[b]{0.48\linewidth}
  \centering
% \centerline{\epsfig{figure=image4.ps,width=4.0cm}}
  \vspace{1.5cm}
  \centerline{(c) Result 4}\medskip
\end{minipage}
%
\caption{Example of placing a figure with experimental results.}
\label{fig:res}
%
\end{figure}


\section{Conclusions}
Conclusions \cite{Lamp86,C2}.
\label{sec:conclusions}

%\subsection{Subheadings}
%\label{ssec:subhead}

%Subheadings should appear in lower case (initial word capitalized) in
%boldface.  They should start at the left margin on a separate line.
 
%\subsubsection{Sub-subheadings}
%\label{sssec:subsubhead}

%Sub-subheadings, as in this paragraph, are discouraged. However, if you
%must use them, they should appear in lower case (initial word
%capitalized) and start at the left margin on a separate line, with paragraph
%text beginning on the following line.  They should be in italics.

% -------------------------------------------------------------------------
% To start a new column (but not a new page) and help balance the last-page
% column length use \vfill\pagebreak.
% -------------------------------------------------------------------------
\vfill
\pagebreak

% References should be produced using the bibtex program from suitable
% BiBTeX files (here: strings, refs, manuals). The IEEEbib.bst bibliography
% style file from IEEE produces unsorted bibliography list.
% -------------------------------------------------------------------------
\bibliographystyle{style/IEEEbib}
\bibliography{bib/refs}

\end{document}
