% Template for IGARSS-2020 paper; to be used with:
%          spconf.sty  - LaTeX style file, and
%          IEEEbib.bst - IEEE bibliography style file.
% --------------------------------------------------------------------------
\documentclass{article}
\usepackage{style/spconf,amsmath,epsfig}

% Example definitions.
% --------------------
\def\x{{\mathbf x}}
\def\L{{\cal L}}

% Title.
% ------
\title{Mapping Riverine Inundation Extents with Graph Signal Filtering of River Depths Determined from Unsupervised Clustering of Synthetic Aperture Radar Imagery}
%
% Single address.
% ---------------
%\name{Author(s) Name(s)\thanks{Thanks to XYZ agency for funding.}}
%\address{Author Affiliation(s)}
%
% For example:
% ------------
%\address{School\\
%	Department\\
%	Address}
%
% Two addresses (uncomment and modify for two-address case).
% ----------------------------------------------------------
\twoauthors
  {Fernando Aristizabal\sthanks{Thanks to National Oceanic and Atmospheric Administration (NOAA) Office of Water Predition (OWP)}}
    {Lynker Technologies \\
    NOAA-OWP Affiliate\\
    National Water Center\\
    205 Hackberry Lane, Tuscaloosa, AL 35401\\
    fernando.aristizabal@noaa.gov}
  {Jasmeet Judge\sthanks{The fourth author contributed while at the University of Florida}}
    {University of Florida\\
    Center for Remote Sensing\\
    Agricultural and Biological Engineering\\
    1741 Museum Rd, Gainesville, FL 32611\\
    jasmeet@ufl.edu}
%
\begin{document}
%\ninept
%
\maketitle
%
\begin{abstract}
abstract
\end{abstract}
%
\begin{keywords}
flood inundation mapping, remote sensing, synthetic aperture radar, unsupervised learning, graph signal processing, height above nearest neighbor 
\end{keywords}
%
\section{Introduction}
\label{sec:intro}

Floods represent one of the most significant natural disasters \cite{national_weather_service_2020,national_weather_service_2019,national_weather_service_2018,us_department_of_commerce_2020} with trends indicating that impacts will only continue to increase in the future \cite{mallakpour2015changing,downton2005reanalysis,kunkel1999temporal,pielke2000precipitation,corringham2019effect,tabari2020climate,milly2002increasing,wing2018estimates}. 
Modern river gage networks like those furnished by the United States Geological Survey (USGS) provide significant utility in the monitoring of river states but still leave considerable amount of data scarcity in terms of spatial sparsity and the lack of observed inundation extents. 
Having more and improved data via remote sensing related to river discharge, stage, and inundation extents could provide key stakeholders with more robust means of forecasting, preparing for, and responding to floods in data scare regions around the world.

A variety of satellite based sensors have been used in the detection of surface water including multi-spectral \cite{nigro2014nasa,sanyal2004application,wang2004using,brakenridge2006modis,jain2005delineation,nghiem2000flood,hussain2011mapping,frazier2000water,dewan2006flood,brivio2002integration}, microwave \cite{de2015global,schumann2015microwave,de2010flood,bindlish2008role,de2009global,kundu2015flood}, and synthetic aperture radar (SAR) \cite{aristizabal2020high,shastry2019using,martinis2010automatic,kudahetty2012flood,schlaffer2015flood,chini2019sentinel,chaabani2018flood,huang2018automated,saatchi2019sar,kasischke2003effects,hess2003dual}.
While SAR does have its own inherent limitations such as terrain distortions and interference with vegetation and anthropogenic features, it boasts a significant number of strengths when compared to other sensors for detection of riverine surface water including high spatial resolution, self-illumination, and low atmospheric attenuation \cite{saatchi2019sar,muckenhuber2016open}.
To counter limitations associated with vegetative and anthropogenic interference, researchers have employed terrain information in the form of digital elevation models (DEM) and detrended DEMs \cite{townsend1998modeling,aristizabal2020high,shastry2019using,saatchi2019sar,twele2016sentinel,huang2017comparison}.
These methods fall short of providing comprehensive solutions to operational inundation mapping with remote sensing in challenging environments.

Here we propose the utilization of unsupervised learning to generate segmented images that will be used for river stage (depth) extraction. These stages are further filtered


\section{Materials and Methods}
\label{sec:materials_and_methods}

The proposed procedure leverages a four step process to counter some of the limitations of SAR based flood inundation mapping (FIM) including errors of omission from vegetation and anthropogenic feature interference and errors of commission including anthropogenic features that cause specular reflections or multi-bounces such as roads or dense buildings.
The first step performs image segmentation on the SAR imagery parameterizing two modes within the data via unsupervised learning.
The following step involves invoking the HAND method to determine the corresponding stage from segmented imagery assuming the mode with the lower backscatter as the inundated class.
The third step takes this noisy water surface plane sampled from the segmented image and HAND data and seeks to smooth the surface via a low-pass graph signal filter utilizing the connectivity defined by the National Hydrography Dataset Version 2.
In closing, these filtered stages are then remapped to inundation extents via the HAND method.

A graphical representation of this procedure is demonstrated in Figure \ref{fig:process_flowchart}.

\begin{figure}[htb]

\begin{minipage}[b]{1.0\linewidth}
  \centering
 \centerline{\epsfig{figure=figures/gsp_flow_chart.jpg,width=8.5cm}}
\end{minipage}

\caption{Flow chart detailing four step procedure for creating remote sensing based flood inundation maps with input, intermediary, and output data.}
\label{fig:process_flowchart}

\end{figure}

\subsection{Validation}
\label{ssec:validation}

The USGS published flood inundation extents for a flood of record taking place during the Hurricane Matthew event on October 2016 in North Carolina, United States. 
In the city of Goldsboro, a water surface manifold created from surveyed high-water marks was intersected with high resolution Lidar-based DEMs to create local flood inundation maps (FIM) \cite{musser2017characterization}.
The map for this event was selected as validation for our proposed approach while common binary classification statistics such as precision, recall, f1-score, and Matthew's correlation coefficient (MCC) \cite{canbek2017binary,chicco2020advantages,matthews1975comparison,baldi2000assessing} were employed to compare the binary predicted and reference maps.

\subsection{Image Segmentation}
\label{ssec:image_segmentation}

Sentinel-1 C-Band SAR data, captured on October 12, 2016, is used in the interferometric wide mode offering vertical-vertical (VV) and vertical-horizontal (VH) polarizations \cite{copernicus2016sentinel} \footnote{S1A\_IW\_GRDH\_1SDV\_20161012T111514 \_20161012T111543\_013456\_01580C\_1783\.SAFE}. 
The SAR product in the two polarizations were calibrated and filtered to remove speckle noise prior to segmentation \cite{zuhlke2015snap,yommy2015sar}. 
Due to SAR's bi-modal distribution, the image is initially segmented via Gaussian Mixture Models (GMM) which assumes there are two components of data each parameterized as Gaussian densities \cite{reynolds2009gaussian,mclachlan2004finite}.
The GMM parameter set comprised of the mean vector, weight vector, and covariance matrix, are learned with the Expectation-Maximization (EM) algorithm. 
Our EM initializes the parameters with those from K-means clusters then uses the probability of membership for each data point to each cluster to derive updated parameters \cite{barazandeh2018behavior,dempster1977maximum}. 
Repeating the process until the log-likelihood of the Gaussian Mixtures converges within some finite tolerance effectively maximizes the local likelihood \cite{barazandeh2018behavior,dempster1977maximum}.
The learned parameters are then used to classify the SAR backscatter values into two distinct classes where the class with lower backscatter intensities tentatively defining the inundated class in the SAR FIM.

\begin{figure}[htb]

\begin{minipage}[b]{1.0\linewidth}
  \centering
 \centerline{\epsfig{figure=figures/histogram_of_usgs_gages_by_order.jpg,width=8.5cm}}
\end{minipage}

\caption{Histogram of stream segments in National Hydrography Dataset Version 2 grouped by stream order and whether it has a United States Geological Survey stream gage or not. Considerable amount of lower order streams are ungaged.}
\label{fig:histogram_gages}

\end{figure}

\subsection{Stage Extraction}
\label{ssec:stage_extraction}

Height Above Nearest Drainage (HAND) is a measure of riverine drainage potentials via normalizing elevations to the nearest relevant drainage line \cite{renno2008hand,nobre2011height,nobre2016hand,aristizabal2020cahaba}. 
Each pixel is assigned a relative elevation value and a catchment assignment referencing the river reach the pixel drains to.
Utilizing the SAR FIM map, the maximum HAND value is extracted per catchment and assigned in equation \ref{eq:stage_extraction} 

\begin{equation}
\label{eq:stage_extraction}
\textbf{S} = max(\textbf{H} \cap \textbf{C}_i)
\end{equation}

where S is a vector of stages, H is a vector of all the HAND values, C is a vector of all catchments, and i indexes the catchments.
These stages are susceptible to considerable noise and demonstrate significant, irrational variations in the water surface profile. 

\subsection{Stage Filtering}
\label{ssec:stage_filtering}

Graph signal processing (GSP) seeks to perform signal processing methods on signals with non-traditional graph structures \cite{gavili2017shift,defferrard2017pygsp,ortega2018graph,chen2014signal}. 
Noisy river stages on networks are an application of GSP filtering due to the dendritic structure of river networks.
The first eigenvalue of an A-B spline wavelet is employed as a low-pass filter to smooth and denoise the extracted stages \cite{defferrard2017pygsp}.

\subsection{HAND Mapping}
\label{ssec:hand_mapping}

The filtered stages are then remapped to inundation extents with the HAND method \cite{renno2008hand,nobre2011height,nobre2016hand,aristizabal2020cahaba}.   

\section{Results and Discussion}
\label{sec:results_and_discussion}



% Below is an example of how to insert images. Delete the ``\vspace'' line,
% uncomment the preceding line ``\centerline...'' and replace ``imageX.ps''
% with a suitable PostScript file name.
%\begin{figure}[htb]

%\begin{minipage}[b]{1.0\linewidth}
%  \centering
%% \centerline{\epsfig{figure=image1.ps,width=8.5cm}}
%  \vspace{2.0cm}
%  \centerline{(a) Result 1}\medskip
%\end{minipage}
%
%\begin{minipage}[b]{.48\linewidth}
%  \centering
%% \centerline{\epsfig{figure=image3.ps,width=4.0cm}}
%%  \vspace{1.5cm}
%  \centerline{(b) Results 3}\medskip
%\end{minipage}
%\hfill
%\begin{minipage}[b]{0.48\linewidth}
%  \centering
%% \centerline{\epsfig{figure=image4.ps,width=4.0cm}}
%  \vspace{1.5cm}
%  \centerline{(c) Result 4}\medskip
%\end{minipage}
%%
%\caption{Example of placing a figure with experimental results.}
%\label{fig:res}
%%
%\end{figure}


\section{Conclusions}
Conclusions
\label{sec:conclusions}

%\subsection{Subheadings}
%\label{ssec:subhead}

%Subheadings should appear in lower case (initial word capitalized) in
%boldface.  They should start at the left margin on a separate line.
 
%\subsubsection{Sub-subheadings}
%\label{sssec:subsubhead}

%Sub-subheadings, as in this paragraph, are discouraged. However, if you
%must use them, they should appear in lower case (initial word
%capitalized) and start at the left margin on a separate line, with paragraph
%text beginning on the following line.  They should be in italics.

% -------------------------------------------------------------------------
% To start a new column (but not a new page) and help balance the last-page
% column length use \vfill\pagebreak.
% -------------------------------------------------------------------------
\vfill
\pagebreak

% References should be produced using the bibtex program from suitable
% BiBTeX files (here: strings, refs, manuals). The IEEEbib.bst bibliography
% style file from IEEE produces unsorted bibliography list.
% -------------------------------------------------------------------------
\bibliographystyle{style/IEEEbib}
\bibliography{bib/refs}

\end{document}
