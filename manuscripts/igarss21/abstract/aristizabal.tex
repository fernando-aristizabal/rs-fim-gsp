% Template for IGARSS-2020 paper; to be used with:
%          spconf.sty  - LaTeX style file, and
%          IEEEbib.bst - IEEE bibliography style file.
% --------------------------------------------------------------------------
\documentclass{article}
\usepackage{style/spconf,amsmath,epsfig}

% Example definitions.
% --------------------
\def\x{{\mathbf x}}
\def\L{{\cal L}}

% Title.
% ------
\title{Riverine Flood Inundation Mapping with Graph Signal Filtering of Stages Determined from Unsupervised Clustering of Synthetic Aperture Rader Imagery}
%
% Single address.
% ---------------
%\name{Author(s) Name(s)\thanks{Thanks to XYZ agency for funding.}}
%\address{Author Affiliation(s)}
%
% For example:
% ------------
%\address{School\\
%	Department\\
%	Address}
%
% Two addresses (uncomment and modify for two-address case).
% ----------------------------------------------------------
\twoauthors
  {Fernando Aristizabal\sthanks{Thanks to National Oceanic and Atmospheric Administration (NOAA) Office of Water Predition (OWP)}}
    {Lynker Technologies \\
    NOAA-OWP Affiliate\\
    National Water Center\\
    205 Hackberry Lane, Tuscaloosa, AL 35401\\
    fernando.aristizabal@noaa.gov}
  {Jasmeet Judge\sthanks{The fourth author contributed while at University of Florida}}
    {University of Florida\\
    Center for Remote Sensing\\
    Agricultural and Biological Engineering\\
    1741 Museum Rd, Gainesville, FL 32611\\
    jasmeet@ufl.edu}
%
\begin{document}
%\ninept
%
\maketitle
%
\begin{abstract}
abstract
\end{abstract}
%
\begin{keywords}
flood inundation mapping, remote sensing, synthetic aperture radar, unsupervised learning, graph signal processing, height above nearest neighbor 
\end{keywords}
%
\section{Introduction}
\label{sec:intro}

Floods represent one of the most significant natural disasters \cite{national_weather_service_2020,national_weather_service_2019,national_weather_service_2018,us_department_of_commerce_2020} with trends indicating that impacts will only continue to increase in the future \cite{mallakpour2015changing,downton2005reanalysis,kunkel1999temporal,pielke2000precipitation,corringham2019effect,tabari2020climate,milly2002increasing,wing2018estimates}. 
While the United States Geological Survey (USGS) operates a network of thousands stream gages across the United States, it still leaves considerable amount of data scarcity particularly along lower order streams as illustrated in Figure \ref{fig:histogram_gages} which compares the distribution of stream orders for gaged and ungaged stream reaches within the National Hydrography Dataset (NHD) Version 2.
Having more and improved data via remote sensing related to river discharge, stage, and inundation extents could provide key stakeholders with more robust means of forecasting, preparing for, and responding to floods in data scare regions around the world.
A variety of satellite based sensors have been used in the detection of surface water including multi-spectral \cite{nigro2014nasa,sanyal2004application,wang2004using,brakenridge2006modis,jain2005delineation,nghiem2000flood,hussain2011mapping,frazier2000water,dewan2006flood,brivio2002integration}, microwave \cite{de2015global,schumann2015microwave,de2010flood,bindlish2008role,de2009global,kundu2015flood}, and synthetic aperture radar (SAR) \cite{aristizabal2020high,shastry2019using,martinis2010automatic,kudahetty2012flood,schlaffer2015flood,chini2019sentinel,chaabani2018flood,huang2018automated,saatchi2019sar,kasischke2003effects,hess2003dual}.
These previous studies also noted limitations with their respective sensors such as interference from clouds, vegetative, and anthropogenic features; radiance distortion from shadows or water color variability; and a dependency of sunlight for multi-spectral sensors.
For active and passive microwave sensors, spatial resolution is limited to typically kilometer scales making detection of flood intelligence moot.
While SAR does have its own inherent limitations such as terrain distortions and interference with vegetation and anthopogenic features, it boasts a significant number of strengths for detection of riverine surface water including high spatial resolution, self-illumination, and low atmospheric attenuation \cite{saatchi2019sar,muckenhuber2016open}.
To counter limitations of remote sensing of flooding, researchers have employed terrain information in the form of digital elevation models (DEM) and detrended DEMs \cite{townsend1998modeling,aristizabal2020high,shastry2019using}

\begin{figure}[htb]

\begin{minipage}[b]{1.0\linewidth}
  \centering
 \centerline{\epsfig{figure=figures/histogram_of_usgs_gages_by_order.jpg,width=8.5cm}}
%  \vspace{2.0cm}
%  \centerline{(a) Result 1}\medskip
\end{minipage}

\caption{Histogram of stream segments in National Hydrography Dataset Version 2 grouped by stream order and whether it has a United States Geological Survey stream gage or not. Considerable amount of lower order streams are ungaged.}
\label{fig:histogram_gages}
%
\end{figure}

\section{Materials and Methods}
\label{sec:materials_and_methods}

Materials and Methods

\section{Results and Discussion}
\label{sec:results_and_discussion}

Results and Discussion

% Below is an example of how to insert images. Delete the ``\vspace'' line,
% uncomment the preceding line ``\centerline...'' and replace ``imageX.ps''
% with a suitable PostScript file name.
\begin{figure}[htb]

\begin{minipage}[b]{1.0\linewidth}
  \centering
% \centerline{\epsfig{figure=image1.ps,width=8.5cm}}
  \vspace{2.0cm}
  \centerline{(a) Result 1}\medskip
\end{minipage}
%
\begin{minipage}[b]{.48\linewidth}
  \centering
% \centerline{\epsfig{figure=image3.ps,width=4.0cm}}
  \vspace{1.5cm}
  \centerline{(b) Results 3}\medskip
\end{minipage}
\hfill
\begin{minipage}[b]{0.48\linewidth}
  \centering
% \centerline{\epsfig{figure=image4.ps,width=4.0cm}}
  \vspace{1.5cm}
  \centerline{(c) Result 4}\medskip
\end{minipage}
%
\caption{Example of placing a figure with experimental results.}
\label{fig:res}
%
\end{figure}


\section{Conclusions}
Conclusions
\label{sec:conclusions}

%\subsection{Subheadings}
%\label{ssec:subhead}

%Subheadings should appear in lower case (initial word capitalized) in
%boldface.  They should start at the left margin on a separate line.
 
%\subsubsection{Sub-subheadings}
%\label{sssec:subsubhead}

%Sub-subheadings, as in this paragraph, are discouraged. However, if you
%must use them, they should appear in lower case (initial word
%capitalized) and start at the left margin on a separate line, with paragraph
%text beginning on the following line.  They should be in italics.

% -------------------------------------------------------------------------
% To start a new column (but not a new page) and help balance the last-page
% column length use \vfill\pagebreak.
% -------------------------------------------------------------------------
\vfill
\pagebreak

% References should be produced using the bibtex program from suitable
% BiBTeX files (here: strings, refs, manuals). The IEEEbib.bst bibliography
% style file from IEEE produces unsorted bibliography list.
% -------------------------------------------------------------------------
\bibliographystyle{style/IEEEbib}
\bibliography{bib/refs}

\end{document}
